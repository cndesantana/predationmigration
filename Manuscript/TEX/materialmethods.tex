\section{Materials and Methods}


\subsection{Dispersal dynamics}

Here we describe the dynamics of predation and migration in explicit
landscapes. There can be several species at each trophic level and
there are several spatially distinct sites. The trophic levels are
resources ($R$), consumers ($H$), and predators ($P$), thus, we have
three distinct metacommunities of resource, consumers, and
predators/parasitoids. To model spatio-temporal changes in the
abundance of these sites, we need to define dispersal and trophic
interaction rules together with population dynamics. We consider two
models of dispersal between sites. Model 1 assumes that dispersal
between any two sites occurs only between the neighboors and it is
density-independent in the sense that dispersal probabilities are a
function of species abundance of the leaving site. Model 1 also
assumes that resources, consumers and predators move independently of
each other. This means individuals are not aware of the state of each
site before the disperse (i.e., the number of prey or predators in
each site). Dispersal to the neighboors is also occuring in the model
2, but now individuals have information of each site and they disperse
with higher probabilities to the sites that have a low number of
predators and a high number of resources available.

% This model can help to simulate how changes of environmental control
% mechanisms can affect those ecological relationships and to describe
% it we will use some descriptions as defined in Table 1.

\subsubsection{Model 1: Dispersal dynamics in non-informed landscapes}

Our first model assumes an individual that emigrates moves with
probability inversely proportional to the distance between the
original and the receiving sites and to the carrying capacity of the
receiving site. Thus, individual movement is independent of the state
of each receiving site. This means individuals do not know the number
of resources or predators before the emigration. This leads to the
dispersal rate of species $k_{\mathcal{\phi}}$ in metacommunity
$\mathcal{\phi}$ from site $j$ to site $i$ (where $\mathcal{\phi}$
stands either for the basal ($R$), intermediate ($I$), or top predator
metacommunity ($P$))

\begin{equation}
  m^{k_{\phi}}_{ij} = \frac{\mathcal{K}^{k_{\phi}}_{i} m_{\phi}}{d_{ij}},
\end{equation}
where $\mathcal{K}^{k_{\phi}}_{i}$, $m_{\phi}$, $d_{ij}$ are the
carrying capacity of the receiving site $i$ of metacommunity $\phi$,
the background emigration rate of metacommunity $\phi$, and the
geographical effective distance between site $j$ and site $i$,
respectively.
 
Given the dispersal rate, the number of individuals of species $k$ in
metacommunity $\phi$ that move from site $j$ to site $i$ is
\begin{equation}
  N^{k_{\phi}}_{ij} =  m^{k_{\phi}}_{ij}  N^{k_{\phi}}_{j},
\end{equation}
where $N^{k_{\phi}}_{j}$ is the abundance of species $k$ in site $j$.

%\begin{equation}
%\vspace{0.25cm}
%\small{M(i,j) = \overbrace{ \left[ \lambda^{i} \frac{ \Delta_{ij}\,f\,\Theta \left(\Delta_{ij}\,f\,\right) }{ \sum_{k \in Neigh\left(i\right)}\Delta_{ik}\,f\,\Theta \left(\Delta_{ik}\,f\,\right) }\right]}^{Biotic} \overbrace{\left[w_{ij} \frac{ \Delta_{ij}\,f_{\eta}\,\Theta \left(\Delta_{ij}\,f_{\eta}\,\right) }{ \sum_{k \in Neigh\left(i\right)}\Delta_{ik}\,f_{\eta}\,\Theta \left(\Delta_{ik}\,f_{\eta}\,\right) }\right]}^{Abiotic}}
%%m_{ij}^{k_{\mathcal{\phi}}} = m_{\mathcal{\phi}} \left(\frac{1}{d_{ij}}\right),
%\label{neutdis}
%\end{equation}
%where... is the ... between site $i$ and $j$ and
%$m_{\mathcal{\phi}}$ is the metacommunity-specific background
%dispersal rate (we need here a figure with the landscape considered
%and we also need to simplify the flow chart and explain this model).

\subsubsection{Model 2: Predator-prey ratio dispesal dynamics in informed landscapes}

Our second model assumes the dispersal rates are density-dependent and
an individual that emigrates has a preference that is a function of
the state of each site. The state of each site is driven by the number
of resources available and the number of predators in each site. Thus,
the migration probability of a species $k_{\mathcal{\phi}}$ in
metacommunity $\mathcal{\phi}$ from site $j$ to site $i$ is defined as
the difference between the number of individuals of species that are
prey and species that are predators of species $k_{\mathcal{\phi}}$ in
site $i$. This leads to the dispersal rate of species
$k_{\mathcal{\phi}}$ in metacommunity $\mathcal{\phi}$ from site $j$
to site $i$

\begin{equation}
  m^{k_{\phi}}_{ij} = \frac{\mathcal{K}^{k_{\phi}}_{i} m_{\phi} \mathcal{M}^{k_{\phi}}_{j}}{d_{ij}},
\end{equation}

\begin{equation}
  m^{k_{\phi}}_{ij} = \frac{\mathcal{K}^{k_{\phi}}_{i} m_{\phi}}{d_{ij}} \mathcal{M}^{k_{\phi}}_{j} \frac{\Delta F^{k\phi}_{ij}}{\sum_{n}^{Neigh(j)}\Delta F^{k\phi}_{nj}},
\end{equation}


where, as in Model 1, $\mathcal{K}^{k_{\phi}}_{i}$, $m_{\phi}$,
$d_{ij}$ are the carrying capacity of the receiving site $i$ of
metacommunity $\phi$, the background emigration rate of metacommunity
$\phi$, and the geographical effective distance between site $j$ and
site $i$, respectively. To take into account informed landscapes we
introduce the concepts of mobility probability and the preference of a species for moving from a site $j$ to a site $i$. 

The mobility probability of species $k$ from site $j$,
$\mathcal{M}^{k_{\phi}}_{j}$, is defined as 1 -
$\frac{r^{k_{\phi}}_{j}}{r^{max_{\phi}}_{j}}$ where $r^{k_{\phi}}_{j}$ and
$r^{max_{\phi}}_{j}$ are the reproductive success of species $k$ and the
maximum reproductive success in metacommunity $\phi$ in site $j$, respectively.
If individuals of species $k$ were able to know their reproductive success in
any site, then they would have information to decide where to go to increase
their reproductive success. For example, if species $k$ reproduces in site $j$
better than in any other site (high $r^{k_{\phi}}_{j}$), then individuals of
species $k$ prefer to stay in site $j$, thus the mobility probability is close
to zero.

The preference of a species $k$ for moving from a site $j$ to a site $i$,
$\Delta F^{k\phi}_{ij}$, is defined as the difference between the preference of
$k$ to stay in $j$, $F^{k\phi}_{j}$, and his preference to stay in $i$,
$F^{k\phi}_{i}$, where the preference of a species $k$ for a site $j$ is given
by the difference in the abundance of prey and the abundance of predators of
$k$ in $j$, $p^{k\phi}_{j} - P^{k\phi}_{j}$.  

SEGUIR DESDE AQUI


Given the dispersal rate, the number of individuals of species $k$ in
metacommunity $\phi$ that move from site $j$ to site $i$ is
\begin{equation}
  N^{k_{\phi}}_{ij} =  m^{k_{\phi}}_{ij}  N^{k_{\phi}}_{j},
\end{equation}
where $N^{k_{\phi}}_{j}$ is the abundance of species $k$ in site $j$.


%EQUATION NOT TRANSFORMED!
%\begin{equation}
%\vspace{0.25cm}
%\small{M(i,j) = \overbrace{ \left[ \lambda^{i} \frac{ \Delta_{ij}\,f\,\Theta \left(\Delta_{ij}\,f\,\right) }{ \sum_{k \in Neigh\left(i\right)}\Delta_{ik}\,f\,\Theta \left(\Delta_{ik}\,f\,\right) }\right]}^{Biotic} \overbrace{\left[w_{ij} \frac{ \Delta_{ij}\,f_{\eta}\,\Theta \left(\Delta_{ij}\,f_{\eta}\,\right) }{ \sum_{k \in Neigh\left(i\right)}\Delta_{ik}\,f_{\eta}\,\Theta \left(\Delta_{ik}\,f_{\eta}\,\right) }\right]}^{Abiotic}}
%\end{equation}
%
%\vspace{0.5cm}
%$$
%\Delta_{ij}\,f = f^{i,j} - f^{j,i}\left\{ \begin{array}{rl}
% f^{i,j} = \rho_{H}(j) + \rho_{P}(i) \\
% f^{j,i} = \rho_{H}(i) + \rho_{P}(j) \\
%       \end{array} \right.
%$$
%
%\vspace{0.5cm}
%\centering $\lambda^{i} = \frac{1}{2} \left( 1 - RE^{i} \right)$ \\
%
%\vspace{0.25cm}
%\centering $ RE^{i} = \frac{ New^{t} }{ N^{t} }$\\
%
%\vspace{0.25cm}
%$$
% \Delta_{ij}\,f_{\eta} = f_{\eta}^{j} - f_{\eta}^{i} \left\{ \begin{array}{rl}
% f_{\eta}^{i} = \eta_{sp}^{*} - \eta_{sp}^{i} \\
% f_{\eta}^{j} = \eta_{sp}^{*} - \eta_{sp}^{i} \\
%       \end{array} \right.
%$$
%
%\vspace{0.25cm}
%$$w_{ij} = Connectivity \: between \: sites \: i and \: j$$

where... is the ... between site $i$ and $j$ and $m_{\mathcal{\phi}}$
is the metacommunity-specific background dispersal rate (we need here
a figure with the landscape considered and we also need to simplify
the flow chart and explain this model). The migration of a species
$k_{\mathcal{\phi}}$ from site $j$ to a neighborhood site $i$ occurs
only if $m_{j}^k(t) > m_{i}^k(t)$. The number of individuals of
species $k$ that move from $i$ to $j$ must respect the threshold
imposed by the carrying capacity of the target patch ($cc_j^k(t)$) and
is defined as seen below:

\subsubsection{Model 3: Predator-prey ratio dispersal dynamics in informed and niche-driven landscapes}



\subsection{Demographic dynamics}

In the previous section we have described the dispersal dynamics. Here
we describe the birth and death probabilities associated with each model...

\subsubsection{Model 1: Dispersal dynamics in non-informed landscapes}

At each time (specify the MC rate) we leave all the individuals of
each site $i$ to be choosen by a \emph{Multinomial Distribution}
\cite{levin1981representation}. The chosen individual $k$ can have
three different behaviors:
\begin{enumerate}
\item it can die for natural reasons;
\item it can eat one individual among its prey; 
\item if $k$ have eaten an individual among its prey, so it can give an offspring. 
\end{enumerate}
If $k$ individual is not a predator (if it is a basal species) the
model assumes it has infinity food suply and the only possible
behaviors are 1. and 2.. For each MC time-step $mc$, this simulation
is repeated for all individuals of each patch of the landscape. The
births will occur only if there is free space in the patch $i$, that
means, if the number of individuals alive at $i$ is lower than its
carrying capacity ($cc$). For each time $t$ in which one individual of
species $k$ gives an offspring in a patch $i$, its number of
individuals in this patch will be increased by 1; for each time $t$ in
which one individual of species $k$ dies naturally or by predation in
a patch $i$, its number of individuals will be decreased by 1. In
Figure \ref{fig:Fluxogram} we show a fluxogram that summarizes the
running of the predation dynamic of the model.

%%%%%%%%%%%%%%%%%%
%%%%%%%%%%%%%%%%%%% B I R T H      P R O B A B I L I T Y
%%%%%%%%%%%%%%%%%%
EQUATION NOT TRANSFORMED!

\vspace{1cm}
\begin{itemize}
\item \textbf{Birth Probability}
\end{itemize}

\begin{align*}
Bp\left(k\right) & = \tikz[baseline]{\node[fill=blue!20,ellipse,anchor=base] (t1) {$\left[1 - \rho\left(k\right)\right]$};} \times \tikz[baseline]{\node[fill=red!20,anchor=base] (t2) {$\left[\sum_{b \in H\left(k\right)}\rho\left(b\right)\left(1-\sum_{c \in P\left(b\right)}\rho\left(c\right)\right)\right]$};} \\
& \times \tikz[baseline]{\node[fill=green!20,anchor=base] (t3) {$\left[1-\sum_{c \in P\left(k\right)}\rho\left(c\right) \right]$};}
\end{align*}

\textbf{Where:} Availability of Resources of Basal Species is 1.0

%%%%%%%%%%%%%%%%%%%%%%%%%%
%%%%%%%%%%%%%%%%%%%%%%%%%%% D E A T H     P R O B A B I L I T Y
%%%%%%%%%%%%%%%%%%%%%%%%%%
EQUATION NOT TRANSFORMED!

\vspace{1cm}
\begin{itemize}
\item \textbf{Death Probability}
\end{itemize}

\begin{align*}
Dp\left(k\right) & = \tikz[baseline]{\node[fill=blue!20,ellipse,anchor=base] (t1) {$\left[\rho\left(k\right)\right]$};} \times \tikz[baseline]{\node[fill=red!20,anchor=base] (t2) {$\left[\sum_{b \in H\left(k\right)}\left(1-\rho\left(b\right)\right)\left(\sum_{c \in P\left(b\right)}\rho\left(c\right)\right)\right]$};} \\ 
& \times \tikz[baseline]{\node[fill=green!20,anchor=base] (t3) {$\left[1-\sum_{c \in P\left(k\right)}\rho\left(c\right) \right]$};} 
\end{align*}

\textbf{Where:} Death Probability of Basal Species is 1.0

%%%%%%%%%%%%%%%%%%%%%%%%%%
%%%%%%%%%%%%%%%%%%%%%%%%%%% N A T U R A L     D E A T H     P R O B A B I L I T Y
%%%%%%%%%%%%%%%%%%%%%%%%%%
EQUATION NOT TRANSFORMED!

\vspace{1cm}
\begin{itemize}
\item \textbf{Natural Death Probability}
\end{itemize}

\begin{align*}
NDp\left(k\right) & = \tikz[baseline]{\node[fill=blue!20,ellipse,anchor=base] (t1) {$\left[\rho\left(k\right)\right]$};} \times \tikz[baseline]{\node[fill=red!20,anchor=base] (t2) {$\left[\sum_{b \in H\left(k\right)}\left(1-\rho\left(b\right)\right)\left(\sum_{c \in P\left(b\right)}\rho\left(c\right)\right)\right]$};} \\ 
& \times \tikz[baseline]{\node[fill=green!20,anchor=base] (t3) {$\left[1-\sum_{c \in P\left(k\right)}\rho\left(c\right) \right]$};} 
\end{align*}

%%%%%%%%%%%%%%%%
%%%%%%%%%%%%%%%% S P E C I E S     C A R R Y I N G     C A P A C I T Y
%%%%%%%%%%%%%%%%
EQUATION NOT TRANSFORMED!

\vspace{1cm}
\begin{itemize}
\item \textbf{Carrying Capacity (for each species in each site)}
\end{itemize}

\begin{align*}
CC\left(k\right) & =  \tikz[baseline]{\node[fill=red!20,ellipse,anchor=base] (t2) {$\left[\sum_{b \in H\left(k\right)}\frac{\left(\rho\left(b\right)\right)}{\left(\sum_{c \in P\left(b\right)}\rho\left(c\right)\right)}\right]$};}
\end{align*}

\subsubsection{Model 2: Predator-prey ratio dispesal dynamics in informed landscapes}

At each time (specify the MC rate) we leave all the individuals of
each site $i$ to be choosen by a \emph{Multinomial Distribution}
\cite{levin1981representation}. The chosen individual $k$ can have
three different behaviors:
\begin{enumerate}
\item it can die for natural reasons;
\item it can eat one individual among its prey; 
\item if $k$ have eaten an individual among its prey, so it can give an offspring. 
\end{enumerate}
If $k$ individual is not a predator (if it is a basal species) the
model assumes it has infinity food suply and the only possible
behaviors are 1. and 2.. For each MC time-step $mc$, this simulation
is repeated for all individuals of each patch of the landscape. The
births will occur only if there is free space in the patch $i$, that
means, if the number of individuals alive at $i$ is lower than its
carrying capacity ($cc$). For each time $t$ in which one individual of
species $k$ gives an offspring in a patch $i$, its number of
individuals in this patch will be increased by 1; for each time $t$ in
which one individual of species $k$ dies naturally or by predation in
a patch $i$, its number of individuals will be decreased by 1. In
Figure \ref{fig:Fluxogram} we show a fluxogram that summarizes the
running of the predation dynamic of the model.

%%%%%%%%%%%%%%%%%%
%%%%%%%%%%%%%%%%%%% B I R T H      P R O B A B I L I T Y
%%%%%%%%%%%%%%%%%%
EQUATION NOT TRANSFORMED!

\vspace{1cm}
\begin{itemize}
\item \textbf{Birth Probability}
\end{itemize}


\begin{align*}
Bp\left(k\right) & = \tikz[baseline]{\node[fill=blue!20,ellipse,anchor=base] (t1) {$\left[1 - \rho\left(k\right)\right]$};} \times \tikz[baseline]{\node[fill=red!20,anchor=base] (t2) {$\left[\sum_{b \in H\left(k\right)}\rho\left(b\right)\left(1-\sum_{c \in P\left(b\right)}\rho\left(c\right)\right)\right]$};} \\
& \times \tikz[baseline]{\node[fill=green!20,anchor=base] (t3) {$\left[1-\sum_{c \in P\left(k\right)}\rho\left(c\right) \right]$};}
\end{align*}

\textbf{Where:} Availability of Resources of Basal Species is 1.0

%%%%%%%%%%%%%%%%%%%%%%%%%%
%%%%%%%%%%%%%%%%%%%%%%%%%%% D E A T H     P R O B A B I L I T Y
%%%%%%%%%%%%%%%%%%%%%%%%%%
EQUATION NOT TRANSFORMED!

\vspace{1cm}
\begin{itemize}
\item \textbf{Death Probability}
\end{itemize}

\begin{align*}
Dp\left(k\right) & = \tikz[baseline]{\node[fill=blue!20,ellipse,anchor=base] (t1) {$\left[\rho\left(k\right)\right]$};} \times \tikz[baseline]{\node[fill=red!20,anchor=base] (t2) {$\left[\sum_{b \in H\left(k\right)}\left(1-\rho\left(b\right)\right)\left(\sum_{c \in P\left(b\right)}\rho\left(c\right)\right)\right]$};} \\ 
& \times \tikz[baseline]{\node[fill=green!20,anchor=base] (t3) {$\left[1-\sum_{c \in P\left(k\right)}\rho\left(c\right) \right]$};} 
\end{align*}

\textbf{Where:} Death Probability of Basal Species is 1.0

%%%%%%%%%%%%%%%%%%%%%%%%%%
%%%%%%%%%%%%%%%%%%%%%%%%%%% N A T U R A L     D E A T H     P R O B A B I L I T Y
%%%%%%%%%%%%%%%%%%%%%%%%%%
EQUATION NOT TRANSFORMED!

\vspace{1cm}
\begin{itemize}
\item \textbf{Natural Death Probability}
\end{itemize}

\begin{align*}
NDp\left(k\right) & = \tikz[baseline]{\node[fill=blue!20,ellipse,anchor=base] (t1) {$\left[\rho\left(k\right)\right]$};} \times \tikz[baseline]{\node[fill=red!20,anchor=base] (t2) {$\left[\sum_{b \in H\left(k\right)}\left(1-\rho\left(b\right)\right)\left(\sum_{c \in P\left(b\right)}\rho\left(c\right)\right)\right]$};} \\ 
& \times \tikz[baseline]{\node[fill=green!20,anchor=base] (t3) {$\left[1-\sum_{c \in P\left(k\right)}\rho\left(c\right) \right]$};} 
\end{align*}

%%%%%%%%%%%%%%%%
%%%%%%%%%%%%%%%% S P E C I E S     C A R R Y I N G     C A P A C I T Y
%%%%%%%%%%%%%%%%
EQUATION NOT TRANSFORMED!

\vspace{1cm}
\begin{itemize}
\item \textbf{Carrying Capacity (for each species in each site)}
\end{itemize}

\begin{align*}
CC\left(k\right) & =  \tikz[baseline]{\node[fill=red!20,ellipse,anchor=base] (t2) {$\left[\sum_{b \in H\left(k\right)}\frac{\left(\rho\left(b\right)\right)}{\left(\sum_{c \in P\left(b\right)}\rho\left(c\right)\right)}\right]$};}
\end{align*}


\subsubsection{Model 3: Predator-prey ratio dispersal dynamics in informed and niche-driven landscapes}


\subsection{Spatial landscapes}

We consider several spatial configurations. We start with a
2-dimensional toroidal lattice where individuals only move to the
nearest 4-neighborhoods.

\subsection{Multi-trophic metacommunity dynamics}

Here, we describe the equations that combine dispersal, trophic and
population dynamics across multiple sites and trophic levels in a
broad geographic region. We track demography and species abundances in
each site by considering at each time step birth-death events across
all the sites.

%The new individual can be of the same species as the dead individual
%or of another species of the same metacommunity. The recruitment is
%either due to local reproduction (birth) or dispersal from another
%sampled site, or immigration from the regional species pool. The
%following equations provide a mathematical description of the
%multi-trophic metacommunity dynamics
EQUATION NOT TRANSFORMED!

\begin{equation}
 \left\{\left[ 1 - NDp\left(k\right) \right]\left[ \sum_{b \in H\left(k\right)}\rho\left(b\right)Dp\left(b\right) \right]\left[Bp\left(k\right)\right] - \left[\sum_{c \in P\left(k\right)}\rho\left(c\right)\left( 1-NDp \left(c \right) \right)\frac{\rho\left(k\right)}{\sum_{d \in H\left(c\right)}\rho\left(d\right)}Dp\left(k\right)\right] - \left[NDp\left(k\right)\right] \right\} 
\end{equation}
where ... describes the probability to choose...

\subsection{Spatial patterns of ecological and spatial networks}


\subsubsection{Simulations}

We use a Monte Carlo (MC) approach to simulate the dispersal and
predation dynamics. Sites are represented as nodes of a geographical
neighborhood network and the connectivity of those sites are
represented as edges of this network (refs). Trophic relationships
among species within each site are represented by a directed network
in which each node represents a species and each directed link
represents a trophic relationship between a pair of species.

Define background emigration rate and the values explored
Define carrying capacity used, equal or different across sites. 


%\subsection{Predation dynamics}

%At each time of the MC simulation we leave all the individuals of each
%patch $i$ to be choosen by a \emph{Multinomial Distribution}
%\cite{levin1981representation}. The chosen individual $k$ can have
%three different behaviors:
%\begin{enumerate}
%\item it can die for natural reasons; 
%\item it can eat one individual among its prey; 
%\item if $k$ have eaten an individual among its prey, so it can give an offspring. 
%\end{enumerate}

%If $k$ individual is not a predator (if it is a basal species) the
%model assumes it has infinity food suply and the only possible
%behaviors are 1. and 2.. For each MC time-step $mc$, this simulation
%is repeated for all individuals of each patch of the landscape. The
%births will occur only if there is free space in the patch $i$, that
%means, if the number of individuals alive at $i$ is lower than its
%carrying capacity ($cc$). For each time $t$ in which one individual of
%species $k$ gives an offspring in a patch $i$, its number of
%individuals in this patch will be increased by 1; for each time $t$ in
%which one individual of species $k$ dies naturally or by predation in
%a patch $i$, its number of individuals will be decreased by 1. In
%Figure \ref{fig:Fluxogram} we show a fluxogram that summarizes the
%running of the predation dynamic of the model.

%\vspace{0.25cm}
%\subsubsection{Predation Equations}

%\vspace{0.25cm}
%\begin{itemize}
%\item \textbf{General Equation}
%\end{itemize}

%\begin{equation}
% \left\{\left[ 1 - NDp\left(k\right) \right]\left[ \sum_{b \in H\left(k\right)}\rho\left(b\right)Dp\left(b\right) \right]\left[Bp\left(k\right)\right] - \left[\sum_{c \in P\left(k\right)}\rho\left(c\right)\left( 1-NDp \left(c %\right) \right)\frac{\rho\left(k\right)}{\sum_{d \in H\left(c\right)}\rho\left(d\right)}Dp\left(k\right)\right] - \left[NDp\left(k\right)\right] \right\} 
%\end{equation}

%%%%%%%%%%%%%%%%%%
%%%%%%%%%%%%%%%%%%% B I R T H      P R O B A B I L I T Y
%%%%%%%%%%%%%%%%%%
%\vspace{1cm}
%\begin{itemize}
%\item \textbf{Self-Organized Parameters: Birth Probability}
%\end{itemize}

%\begin{align*}
%Bp\left(k\right) & = \tikz[baseline]{\node[fill=blue!20,ellipse,anchor=base] (t1) {$\left[1 - \rho\left(k\right)\right]$};} \times \tikz[baseline]{\node[fill=red!20,anchor=base] (t2) {$\left[\sum_{b \in H\left(k\right)}\rho\left(b\right)\left(1-\sum_{c \in P\left(b\right)}\rho\left(c\right)\right)\right]$};} \\
%& \times \tikz[baseline]{\node[fill=green!20,anchor=base] (t3) {$\left[1-\sum_{c \in P\left(k\right)}\rho\left(c\right) \right]$};}
%\end{align*}

%\textbf{Where:} Availability of Resources of Basal Species is 1.0

%%%%%%%%%%%%%%%%%%%%%%%%%%
%%%%%%%%%%%%%%%%%%%%%%%%%%% D E A T H     P R O B A B I L I T Y
%%%%%%%%%%%%%%%%%%%%%%%%%%
%\vspace{1cm}
%\begin{itemize}
%\item \textbf{Self-Organized Parameters: Death Probability}
%\end{itemize}

%\begin{align*}
%Dp\left(k\right) & = \tikz[baseline]{\node[fill=blue!20,ellipse,anchor=base] (t1) {$\left[\rho\left(k\right)\right]$};} \times \tikz[baseline]{\node[fill=red!20,anchor=base] (t2) {$\left[\sum_{b \in H\left(k\right)}\left(1-\rh%o\left(b\right)\right)\left(\sum_{c \in P\left(b\right)}\rho\left(c\right)\right)\right]$};} \\ 
%& \times \tikz[baseline]{\node[fill=green!20,anchor=base] (t3) {$\left[1-\sum_{c \in P\left(k\right)}\rho\left(c\right) \right]$};} 
%\end{align*}

%\textbf{Where:} Death Probability of Basal Species is 1.0

%%%%%%%%%%%%%%%%%%%%%%%%%%
%%%%%%%%%%%%%%%%%%%%%%%%%%% N A T U R A L     D E A T H     P R O B A B I L I T Y
%%%%%%%%%%%%%%%%%%%%%%%%%%
%\vspace{1cm}
%\begin{itemize}
%\item \textbf{Self-Organized Parameters: Natural Death Probability}
%\end{itemize}

%\begin{align*}
%NDp\left(k\right) & = \tikz[baseline]{\node[fill=blue!20,ellipse,anchor=base] (t1) {$\left[\rho\left(k\right)\right]$};} \times \tikz[baseline]{\node[fill=red!20,anchor=base] (t2) {$\left[\sum_{b \in H\left(k\right)}\left(1-\r%ho\left(b\right)\right)\left(\sum_{c \in P\left(b\right)}\rho\left(c\right)\right)\right]$};} \\ 
%& \times \tikz[baseline]{\node[fill=green!20,anchor=base] (t3) {$\left[1-\sum_{c \in P\left(k\right)}\rho\left(c\right) \right]$};} 
%\end{align*}

%%%%%%%%%%%%%%%%
%%%%%%%%%%%%%%%% S P E C I E S     C A R R Y I N G     C A P A C I T Y
%%%%%%%%%%%%%%%%
%\vspace{1cm}
%\begin{itemize}
%\item \textbf{Self-Organized Parameters: Carrying Capacity (for each species in each site)}
%\end{itemize}

%\begin{align*}
%CC\left(k\right) & =  \tikz[baseline]{\node[fill=red!20,ellipse,anchor=base] (t2) {$\left[\sum_{b \in H\left(k\right)}\frac{\left(\rho\left(b\right)\right)}{\left(\sum_{c \in P\left(b\right)}\rho\left(c\right)\right)}\right]$}%;}
%\end{align*}

%%%%%%%%%%%%%%%%
%%%%%%%%%%%%%%%% M O B I L I T Y 
%%%%%%%%%%%%%%%% 

%\subsection{Dispersal Migration}

%The model allows the Dispersal Migration of individuals living in any
%patche $i$ to each patch in its neighborhood.  Is
%assumed that, for each time step, each species in the landscape has a
%level of preference to each patch. The preference of a species $k$ for
%a patch $i$ in time $t$ ($Pref_{i}^k(t)$) is defined as the difference
%between the number of individuals of species that are prey and species
%that are predators of $k$ in that patch. The migration of a species
%$k$ from a patch $i$ to a neighborhood patch $j$ will occur only if
%$Pref_{j}^k(t) > Pref_{i}^k(t)$. The number of individuals of species
%$k$ that move from $i$ to $j$ must respect the threshold imposed by
%the carrying capacity of the target patch ($cc_j^k(t)$) and is defined
%as seen below:

%\vspace{0.25cm}
%\subsubsection{Migration Equations}

%%%%%%\begin{frame}
%%%%%%\frametitle{Mobility Dynamics}
%%%%%%%\includegraphics[width=1.0\textwidth]{./mobility.eps}
%%%%%%\end{frame}
%%%%%%

%\vspace{1cm}
%\begin{itemize}
%    \item \textbf{Mobility of species \emph{sp} from site \emph{i} to \emph{j}} 
%\end{itemize}

%$\Delta N_{sp}\left(i\right) = \sum_{j \in Neigh\left(i\right)} \left( \left( N_{sp}(j) \; M_{sp}(j,i) \; - \; \tikz[baseline]{\node[fill=blue!20,anchor=base] (t1) {$N_{sp}(i) $};} \tikz[baseline]{\node[fill=red!20,anchor=base%] (t2) {$M_{sp}(i,j)$};} \right) \right)$


%\vspace{0.25cm} \small{$M(i,j) = \overbrace{ \left[ \lambda^{i} \frac{
%        \Delta_{ij}\,f\,\Theta \left(\Delta_{ij}\,f\,\right) }{
%        \sum_{k \in Neigh\left(i\right)}\Delta_{ik}\,f\,\Theta
%        \left(\Delta_{ik}\,f\,\right) }\right]}^{Biotic}
%  \overbrace{\left[w_{ij} \frac{ \Delta_{ij}\,f_{\eta}\,\Theta
%        \left(\Delta_{ij}\,f_{\eta}\,\right) }{ \sum_{k \in
%          Neigh\left(i\right)}\Delta_{ik}\,f_{\eta}\,\Theta
%        \left(\Delta_{ik}\,f_{\eta}\,\right) }\right]}^{Abiotic}$}

%\vspace{0.5cm}
%$$
%\Delta_{ij}\,f = f^{i,j} - f^{j,i}\left\{ \begin{array}{rl}
% f^{i,j} = \rho_{H}(j) + \rho_{P}(i) \\
% f^{j,i} = \rho_{H}(i) + \rho_{P}(j) \\
%       \end{array} \right.
%$$

%\vspace{0.5cm}
%\centering $\lambda^{i} = \frac{1}{2} \left( 1 - RE^{i} \right)$ \\

%\vspace{0.25cm}
%\centering $ RE^{i} = \frac{ New^{t} }{ N^{t} }$

%\vspace{0.25cm}
%$$
% \Delta_{ij}\,f_{\eta} = f_{\eta}^{j} - f_{\eta}^{i} \left\{ \begin{array}{rl}
% f_{\eta}^{i} = \eta_{sp}^{*} - \eta_{sp}^{i} \\
% f_{\eta}^{j} = \eta_{sp}^{*} - \eta_{sp}^{i} \\
%       \end{array} \right.
%$$

%\vspace{0.25cm}
%$$w_{ij} = Connectivity \: between \: sites \: i and \: j$$

%%%%%%%%%%%%%%%%%%%%%% 
%%%%%%%%%%%%%%%%%%%%%%

%\vspace{1cm}
% Do NOT remove this, even if you are not including acknowledgments

